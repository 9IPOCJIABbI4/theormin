
\documentclass[a4paper,12pt]{article}

% Этот шаблон документа разработан в 2014 году
% Данилом Фёдоровых (danil@fedorovykh.ru) 
% для использования в курсе 
% <<Документы и презентации в \LaTeX>>, записанном НИУ ВШЭ
% для Coursera.org: http://coursera.org/course/latex .
% Исходная версия шаблона --- 
% https://www.writelatex.com/coursera/latex/5.3

% В этом документе преамбула

%%% Работа с русским языком
\usepackage{cmap}					% поиск в PDF
\usepackage{mathtext} 				% русские буквы в формулах
\usepackage[T2A]{fontenc}			% кодировка
\usepackage[utf8x]{inputenc}			% кодировка исходного текста
\usepackage[english,russian]{babel}	% локализация и переносы
\usepackage{indentfirst}
\frenchspacing

\renewcommand{\epsilon}{\ensuremath{\varepsilon}}
\renewcommand{\phi}{\ensuremath{\varphi}}
\renewcommand{\kappa}{\ensuremath{\varkappa}}
\renewcommand{\le}{\ensuremath{\leqslant}}
\renewcommand{\leq}{\ensuremath{\leqslant}}
\renewcommand{\ge}{\ensuremath{\geqslant}}
\renewcommand{\geq}{\ensuremath{\geqslant}}
\renewcommand{\emptyset}{\varnothing}

%%% Дополнительная работа с математикой
\usepackage{amsmath,amsfonts,amssymb,amsthm,mathtools} % AMS
\usepackage{icomma} % "Умная" запятая: $0,2$ --- число, $0, 2$ --- перечисление

%% Номера формул
%\mathtoolsset{showonlyrefs=true} % Показывать номера только у тех формул, на которые есть \eqref{} в тексте.
%\usepackage{leqno} % Нумереация формул слева

%% Свои команды
\DeclareMathOperator{\sgn}{\mathop{sgn}}

%% Перенос знаков в формулах (по Львовскому)
\newcommand*{\hm}[1]{#1\nobreak\discretionary{}
{\hbox{$\mathsurround=0pt #1$}}{}}

%%% Работа с картинками
\usepackage{graphicx}  % Для вставки рисунков
\graphicspath{{figures/}}  % папки с картинками
\setlength\fboxsep{3pt} % Отступ рамки \fbox{} от рисунка
\setlength\fboxrule{1pt} % Толщина линий рамки \fbox{}
\usepackage{wrapfig} % Обтекание рисунков текстом

%%% Работа с таблицами
\usepackage{array,tabularx,tabulary,booktabs} % Дополнительная работа с таблицами
\usepackage{longtable}  % Длинные таблицы
\usepackage{multirow} % Слияние строк в таблице

%%% Теоремы
\theoremstyle{plain} % Это стиль по умолчанию, его можно не переопределять.
\newtheorem{theorem}{Теорема}
\newtheorem*{thm}{Теорема}
\newtheorem{proposition}[theorem]{Утверждение}
 
\theoremstyle{definition} % "Определение"
\newtheorem{corollary}{Следствие}[theorem]
\newtheorem*{dfn}{Определение}
\newtheorem{problem}{Задача}
\newtheorem*{problem*}{Задача}

 
\theoremstyle{remark} % "Примечание"
\newtheorem*{nonum}{Решение}

%%% Программирование
\usepackage{etoolbox} % логические операторы

%%% Страница
%\usepackage{extsizes} % Возможность сделать 14-й шрифт
%\usepackage{geometry} % Простой способ задавать поля
%	\geometry{top=25mm}
%	\geometry{bottom=35mm}
%	\geometry{left=35mm}
%	\geometry{right=20mm}
 
\usepackage{fancyhdr} % Колонтитулы
%	\pagestyle{fancy}
 %	\renewcommand{\headrulewidth}{0pt}  % Толщина линейки, отчеркивающей верхний колонтитул
	%\lfoot{Нижний левый}
	%\rfoot{Нижний правый}
	%\rhead{Верхний правый}
	%\chead{Верхний в центре}
	%\lhead{Верхний левый}
	%\cfoot{Нижний в центре} % По умолчанию здесь номер страницы

\usepackage{setspace} % Интерлиньяж
%\onehalfspacing % Интерлиньяж 1.5
%\doublespacing % Интерлиньяж 2
%\singlespacing % Интерлиньяж 1

\usepackage{lastpage} % Узнать, сколько всего страниц в документе.

\usepackage{soul} % Модификаторы начертания

\usepackage{hyperref}
\usepackage[usenames,dvipsnames,svgnames,table,rgb,pdftex,dvipsnames]{xcolor}
\hypersetup{				% Гиперссылки
    unicode=true,           % русские буквы в раздела PDF
    pdftitle={Заголовок},   % Заголовок
    pdfauthor={Автор},      % Автор
    pdfsubject={Тема},      % Тема
    pdfcreator={Создатель}, % Создатель
    pdfproducer={Производитель}, % Производитель
    pdfkeywords={keyword1} {key2} {key3}, % Ключевые слова
    colorlinks=true,       	% false: ссылки в рамках; true: цветные ссылки
    linkcolor=red,          % внутренние ссылки
    citecolor=black,        % на библиографию
    filecolor=magenta,      % на файлы
    urlcolor=cyan           % на URL
}

\usepackage{csquotes} % Еще инструменты для ссылок

%\usepackage[style=apa,maxcitenames=2,backend=biber,sorting=nty]{biblatex}

\usepackage{multicol} % Несколько колонок

\usepackage{tikz} % Работа с графикой
\usepackage{pgfplots}
\usepackage{pgfplotstable}
\usepackage{coloremoji}
\usepackage{floatrow}
\usepackage{subcaption}
\newcommand*{\N}{\mathbb{N}}
\newcommand*{\R}{\mathbb{R}}
\newcommand*{\K}{\mathbb{K}}
\newcommand*{\V}{\mathcal{V}}
\newcommand*{\A}{\mathcal{A}}
\newcommand*{\ii}{\mathbf{1}}
\newcommand*{\oo}{\mathbf{0}}
\newcommand*{\ba}{\mathbf{a}}
\newcommand*{\bb}{\mathbf{b}}
\newcommand*{\Q}{\mathbb{Q}}
\graphicspath{{images/}}
%\usepackage{breqn}

\renewcommand\thesubfigure{\asbuk{subfigure}}
%\addbibresource{master.bib}

\usepackage{import}
\usepackage{pdfpages}
\usepackage{transparent}
\usepackage{xcolor}
\usepackage{xifthen}

%\newcommand{\incfig}[1]{%
%    \def\svgwidth{\columnwidth}
%    \import{./figures/}{#1.pdf_tex}
%}


\newcommand{\incfig}[2][1]{%
    \def\svgwidth{#1\columnwidth}
    \import{./figures/}{#2.pdf_tex}
}
\usepackage{titlesec}
\titleformat{\section}{\normalfont\Large\bfseries}{}{0pt}{}


\pdfsuppresswarningpagegroup=1
\pgfplotsset{compat=1.16}
\renewcommand{\thesection}{}
\renewcommand{\thesubsection}{\arabic{subsection}}

\usepackage{marginnote}
\usepackage{xargs}
\usepackage[colorinlistoftodos,prependcaption,shadow, textwidth=3cm]{todonotes}
\newcommandx{\improvement}[2][1=]{\todo[disable,linecolor=OliveGreen,backgroundcolor=OliveGreen!25,bordercolor=OliveGreen,#1]{#2}}
\newcommandx{\unsure}[2][1=]{\todo[disable,linecolor=YellowOrange,backgroundcolor=YellowOrange!25,bordercolor=YellowOrange,#1]{#2}}
\newcommandx{\didntdone}[2][1=]{\todo[disable,linecolor=red,backgroundcolor=red!25,bordercolor=red,#1]{#2}}
\newcommandx{\impr}[2][1=]{\todo[linecolor=OliveGreen,backgroundcolor=OliveGreen!25,bordercolor=OliveGreen,#1]{#2}}
%%\setcounter{secnumdepth}{0}
%%% Local Variables:
%%% mode: latex
%%% TeX-master: "master"
%%% End:


\title{Теорминимум
по теормеху}
\author{Драчов Ярослав\\ Факультет общей и прикладной физики, МФТИ}


\begin{document} % конец преамбулы, начало документа
%\input{titlepage}
%\pdfsuppresswarningpagegroup=1a
\maketitle
\section*{Первое задание}
\subsection{Определение положения равновесия}
\begin{dfn}
	Некоторое положение системы тогда и только тогда является её
	\emph{положением
	равновесия}, когда в этом положении все обобщённые силы равны нулю:
	\[
		Q_i=-\frac{\partial \Pi}{\partial q_i} =0, \quad (i=1,\,2,\ldots
		,\,n),
	\] 
	где $\Pi$ --- потенциальная энергия системы, которая в случае
	консервативной системы явно от времени не зависит.
\end{dfn}
\subsection{Устойчивое положение равновесия}
\begin{dfn}
	Положение равновесия $q_1=q_2=\ldots=0$ называется  \emph{устойчивым},
	если для любого $\epsilon >0$ существует такое $\delta=\delta(\epsilon)$,
	что для всех $t>t_0$ выполняются неравенства
	\[
		|q_i(t)|<\epsilon,\quad |\dot{q}_i(t)<\epsilon \quad(i=1,\,2,\ldots
		,\,n)
	\] 
	при условии, что в начальный момент $t=t_0$ 
	\[
		|q_i(t_0)|<\delta,\quad |\dot{q}_i(t_0)<\delta|
	.\] 
\end{dfn}
\subsection{Теорема Лагранжа-Дирихле}
\begin{thm}[Лагранжа-Дирихле]
	Если в положении равновесия консервативной системы потенциальная 
	энергия имеет строгий  локальный минимум,  то это положение равновесия
	устойчиво.
\end{thm}
\subsection{Первая теорема Ляпунова о неустойчивости}
\begin{thm}[Ляпунова, 1-я]
Если потенцмальная энергия консервативной системы в положении равновесия не
имеет минимума и это узнаётся уже по членам второго порядка в разложении
функции $\Pi$ в ряд в окрестности положения равновесия без необходимости
рассматривания членов высших порядков, то положение равновесия неустойчиво.
\end{thm}
\subsection{Вторая теорема Ляпунова о неустойчивости}
\begin{thm}[Ляпунова, 2-я]
	Если в положении равновесия потенциальная энергия имеет максимум и
	это узнаётся по членам наименее высокого порядка, которые действительно
	присутствуют в разложении этой функции в ряд в окрестности положения
	равновесия, то это положение равновесия неустойчиво.
\end{thm}
\subsection{Нормальные координаты}
\begin{dfn}
	Обобщённые координаты $\theta_j$, в которых кинетическая и потенциальная
	энергия системы имеют вид
	\[
		T=\frac{1}{2} \sum_{j=1}^{n} \dot{\theta}_j^2, \qquad
		\Pi=\frac{1}{2} \sum_{j=1}^{n} \lambda_j \theta_j^2,
	\]
	называются \emph{нормальными}.
\end{dfn}
В нормальных координатах лианеризованные уравнения движения в окрестности
положения равновесия имеют вид $n$ не связанных друг с другом уравнений
второго порядка
\[
	\ddot{\theta}_j+\lambda_j \theta_j =0, \quad (j =1,\,2,\ldots,\,n)
.\] 
\subsection{Асимптотически устойчивое положение равновесия}
\begin{dfn}
	Положение равновесия $q_j^0$ называется асимптотически устойчивым,
	если оно устойчиво и, если, кроме того, существует такая 
	$\delta$-окрестность точки $q_j=q_j^0$, $\dot{q}_j=0$ $(j=1,\ldots,\,n)$,
	что для всех $|q_j^0-q_j|<\delta$, $|\dot{q}_j(0)|< \delta$ выполняются
	условия
	\[
		\lim_{t \to \infty} q_j (t)=q_j^0, \quad \lim_{t \to \infty} 
		\dot{q}_j(t)=0,\quad (j=1,\ldots,\,n)
	.\] 
\end{dfn}
\subsection{Теорема Ляпунова о лианеризованных системах}
\begin{thm}[Ляпунова о  лианеризованных системах]
	Если все корни характеристического уравнения
\[
	\det \|\mathbf{A}\lambda^2+\mathbf{B}^* \lambda +(\mathbf{C}+
	\mathbf{C}^*)\|=0
\]
системы дифференциальных уравнений линейного приближения
\[
	\mathbf{A}\ddot{\mathbf{q}}+\mathbf{B}^* \dot{\mathbf{q}}+
	(\mathbf{C}+\mathbf{C}^*)\mathbf{q}=0
\]
имеют отрицательные действительные части, то положение равновесия  $q=0$
исходной системы, описываемой уравнениями
\[
	\frac{d}{dt} \frac{\partial T}{\partial \dot{q}_j} -\frac{\partial T}{\partial q_j} =
	Q_j,\qquad Q_j=-\frac{\partial V}{\partial q_j} +Q_j^* \quad (j=1,\ldots,
	\,n)
,\] 
асимтотически устойчиво. Если хотя бы один корень характеристического уравнения
имеет действительную част, то положение равновесия, определяемое системой,
неустойчиво.
\end{thm}
\subsection{Критерии Рауса-Гурвица и Льенара-Шипара устойчивости многочлена}
\begin{dfn}
	 Назовём \emph{матрицей Гурвица} квадратную  матрицу $m$-го порядка
	 \[
		 \begin{pmatrix} a_1 & a_3 & a_5 & \ldots & 0\\
		 a_0 & a_2 & a_4 & \ldots & 0 \\
	 0 & a_1 & a_3 & \ldots & 0\\
 0 & a_0 & a_2 & \ldots & 0 \\
  & & & \ddots & \vdots\\
  & & & & a_m
 \end{pmatrix} 
	  .\] 
\end{dfn}
Составим главные миноры  матрицы Гурвица (\emph{определители Гурвица})
\[
	\Delta_1=a_1, \; \Delta_2 = \begin{pmatrix} a_1 & a_3 \\ a_0 & a_2 \end{pmatrix},
	\; \Delta_3= \begin{pmatrix} a_1 & a_3 & a_5 \\ a_0 & a_2 & a_4 \\
	0 & a_1 & a_3\end{pmatrix} ,\ldots,\; \Delta_m =a_m \Delta_{m-1}
.\] 
\begin{thm}[Критерий Рауса-Гурвица в форме Льенара-Шипара]
	Для того чтобы все корни уравнения
	\[
	a_0 \lambda^m+a_1 \lambda^{m-1}+\ldots+a_{m-1}\lambda+a_m=0
	\]
	с вещественнными коэффициентами и положительным старшим  коэффициентом
	$a_0$ имели отрицательные вещественные   части, необходимо и достаточно,
	чтобы выполнялись неравенства
	\[
	\Delta_1>0,\quad\Delta_2>0,\quad \ldots, \quad \Delta_m >0
	.\] 
\end{thm}
\subsection{Теорема Ляпунова об\\
	устойчивости/асимптотической устойчивости\\
(функция Ляпунова)}
\begin{thm}[Ляпунова, об устойчивости движения]
	Если дифференциальные уравнения возмущённого движения таковы, что
	существует знакоопределённая функция $V$, производная которой $\dot{V}$ 
	в силу этих уравнений является  или  знакопостоянной функцией
	противоположного знака с $V$, или тождественно равной нулю, то
	невозмущённое движение устойчиво.
\end{thm}
\begin{thm}[Ляпунова, об асимптотической устойчивости]
	Если дифференциальные уравнения возмущённого движения таковы, что
	существует знакоопределённая функция $V(x_1,\,x_2,\ldots,\,x_m)$,
	производная которой $\dot{V}$  в силу этих уравнений  есть
	знакоопределённая функция противоположного знака с $V$, то
	невозмущённое движение асимптотически устойчиво.
\end{thm}
\subsection{Теорема Барбашина-Красовского}
\begin{thm}[Барбашина-Красовского]
	Если для дифференциальных уравнений возмущённого движения можно найти
	положительно определённую функцию $V(y)$, производная которой,
	вычисленная в силу этих уравнений $\dot{V}(y)<0$ вне $K$ и
	$\dot{V}(y)=0$ на $K$, где $K$  --- многообразие точек, содержащее
	единственное решение $y(t) \equiv 0$, то невозмущённое движение
	асимптотически устойчиво, если функция $V(y)$ отрицательно определена,
	то невозмущённое движение асимптотически неустойчиво.
\end{thm}
\subsection{Теорема Четаева о неустойчивости}
\begin{thm}[Четаева, о неустойчивости]
	Если дифференциальные уравнения возмущённого движения таковы, что
	существует функция\\$V(x_1,\,x_2,\ldots,\,x_m)$
	такая, что в сколь угодно
	малой  окрестности
	 \[
		 |x_i|<h \quad (i=1,\,2,\ldots,\,m)
	\] 
	существует область $V>0$ и во всех точках области $V>0$ производная
	$\dot{V}$ в силу этих уравнений принимает положительные значения, то
	невозмущённое движение неустойчиво.
\end{thm}
\subsection{Понятие о бифуркации положений равновесия}
Дифференциальные уравнения динамических систем часто зависят не только от
фазовых переменных, но и от некоторых параметров.
%В  этом случае уравнения
%имеют вид
%\[
%	\dot{\mathbf{x}} = \mathbf{f}(\mathbf{x},\,\boldsymbol{\alpha}),\quad
%	\mathbf{x} \in \R^n, \quad \boldsymbol{\alpha} \in \R^m
%,\] 
%где $\mathbf{x}$ --- фазовый вектор, $\boldsymbol{\alpha}$ --- набор параметров.
Иногда изменение параметров приводит к качественным перестройкам  структуры
фазовых траекторий системы --- изменениям количества положений равновесия,
характера их устойчивости, кардинальным трансформациям траекторий и т.\;п.
В окрестности определённых значений параметров перестройки происходят при
сколь угодно малом изменении параметров и называются \emph{бифуркациями}.
\subsection{Бифуркация Андронова-Хопфа}
\emph{Бифуркацией Андронова-Хопфа} называется бифуркация рождения цикла.
\subsection{Метод Биркгофа(?) приведения к нормальной форме. Понятие резонанса}
\begin{dfn}
	\emph{Нормальной формой} системы
	\[
		\dot{\mathbf{x}}=\mathbf{f}(\mathbf{x}),\quad
		\mathbf{x}=[x_1,\ldots,\,x_n]^T, \quad 
		\mathbf{f}=[f_1,\ldots,\,f_n]^T
	\]
	называется форма содержащая лишь линейные и резонансные слагаемые.
\end{dfn}
При помощи разложения в ряд Тейлора функции   \mathbf{f} в окрестности
положения равновесия выделим в данной системе линейные слагаемые:
\begin{gather*}
	\dot{\mathbf{x}}= \mathbf{Ax}+\boldsymbol{g}(\mathbf{x}),\quad
	\boldsymbol{g}=[g_1,\ldots,\,g_n]^T,\\ \quad g_i(\mathbf{x})=
	\sum_{k_1,\ldots,\,k_n}^{} g_{k_1,\ldots,\,k_n}^i x_1^{k_1}\ldots
	x_n ^{k_n},\quad k_1+\ldots+k_n \ge 2
.\end{gather*}
Здесь $\mathbf{A}$ --- матрица с постоянными коэффициентами размера $n \times n$,
$g_{k_1,\ldots,\,k_n}$ --- постоянные коэффициенты в полиномах $\boldsymbol{g}$.
\begin{dfn}
	\emph{Нормализующим преобразованием} (вплоть до степени $k$) 
	называется последовательность преобразований
	\begin{gather*}
		\tilde{\mathbf{x}}=\mathbf{y}+ \mathbf{p}(\mathbf{y}),\quad
		\mathbf{p}(\mathbf{y})=[p_1,\ldots,\,p_n]^T,\\
		p_i(\mathbf{y})= \sum_{k_1,\ldots,\,k_n}^{} p^i_{k_1,\ldots,\,k_n}
		y_1^{k_1}\ldots y_n^{k_n},\quad k_1+\ldots+k_n=k
	\end{gather*}
	с полиномиальными коэффициентами
	\[
	p_{k_1,\ldots,\,k_n}^i=\frac{g_{k_1,\ldots,\,k_n}^i}{k_1 \lambda_1+
	\ldots+k_n \lambda_n-\lambda_i}
	,\] 
	приводящее систему к её нормальной форме (вплоть до слагаемых степени
	$k$).
\end{dfn}
\begin{dfn}
	Нелинейные слагаемые системы, показатели $k_1,\ldots,\,k_n$ которых
	таковы, что выполняется $\lambda_i=k_1 \lambda_1+\ldots k_n \lambda_n$,
	называются \emph{резонансными}.
\end{dfn}
В соответствии с описанным выше алгоритмом все нерезонансные слагаемые могут
быть исключены из правой части системы при помощи последовательно применяемых
полиномиальных замен. В то же время резонансные слагаемые не могут быть ни
исключены, ни каким-либо образом ими преобразованы.
\subsection{Вынужденные колебания под действием периодической силы. Частотная
характеристика, амплитудно-фазовая характеристика. Необходимые и достаточные
условия возникновения резонанса в таких системах.}
\begin{dfn}
	Колебания, которые возникают благодаря наличию вынуждающей силы,
	зависящей явно от времени, и к которым в пределе стремится суммарное
	движение, назывют \emph{вынужденными колебаниями}.
\end{dfn}
Представим уравнения линейного приближения стационарной системы в виде
\[
	\sum_{k=1}^{n} (a_{jk}\ddot{q}_k+b_{jk}\dot{q}_k+c_{jk}q_k)=Q_j (t)
	\quad (j=1,\ldots,\,n),
\] 
где $Q_j (t)$ --- зависящие явно от времени части обобщённых сил.
Предполагается, что в процессе движения они остаются малыми по модулю и не
выводят систему из малой окрестности положения равновесия.

В связи с тем, что данная система уравнений является линейной, а для линейных
систем имеет место принцип суперпозиции, можно рассмотреть движение системы
под действием какой-либо одной силы из $Q_j(t)$ $(j=1,\ldots,\,n)$, предположив,
что все остальные равны нулю. Определив порознь движения, возникающие под
действием каждой из таких обобщённых сил, их следует затем сложить.

Учитывая это обстоятельство, положим
\[
	Q_2(t)=Q_3(t)=\ldots=Q_n(t)=0,
\]
т.\:е. будем считать, что отлична от нуля только обобщённая сила $Q_1(t)$,
относящаяся к первой обобщённой координате, а все остальные обобщённые силы
такого рода равны нулю.
 
Введём обозначения
\begin{gather*}
	d_{jk}(i\Omega)=a_{jk}(i\Omega)^2+b_{jk}(i\Omega)+c_{jk},\\
	\Delta=\det \|d_{jk}(i\Omega)\|=\begin{pmatrix} d_{11}(i\Omega) &
	\cdots & d_{1n}(i\Omega) \\ \vdots & & \vdots \\ d_{n1}(i\Omega) &
\cdots & d_{nn}(i\Omega)\end{pmatrix} ,\\
W_{1k}(i\Omega)=\frac{\Delta_{1k}}{\Delta},
\end{gather*}
где  $\Delta_{1k}$ --- алгебраическое дополнение расположенного в первой строке
и $k$-ом столбце  элемента определителя $\Delta$.
Тогда при поиске частного решения неоднородной системы дифференциальных уравнений
в виде $\tilde{q}_f=B e ^{i\Omega t}$ и, выделяя в последствии мнимую часть,
получим
\[
	q_k=A |W_{1k}(i\Omega)|\sin[\Omega t +\arg W_{1k}(i\Omega)]
	\quad (k=1,\ldots,\,n)
.\]
\begin{dfn}
	Введённая выше функция $W_{1k}(i\Omega)$ называется \emph{частотной
	характеристикой системы}, или, как говорят иногда, её
	\emph{амлитудно-фазовой
характеристикой}.
\end{dfn}
\begin{dfn}
	 Если отдельно рассмотреть изменение модуля и аргумента вектора
	$W(i\Omega)$ в зависимости от $\Omega$, то получатся характеристики,
	которые называются соответственно \emph{амплитудной} и \emph{фазовой}
	характеристиками системы.
\end{dfn}
\begin{dfn}
	Если амплитудная характеристика системы при некотором значении $\Omega=
	\Omega^*$ 
	имеет отчётливо выраженный пик, то при одной и той же амплитуде
	внешней силы $Q_1$ амплитуда отклика резко возрастает, когда частота
	внешней силы  приближается к значению $\Omega=\Omega^*$. Это явление
	называют \emph{резонансом}.
\end{dfn}
\section*{Второе задание}
\setcounter{subsection}{0}
\subsection{Понятие краевой задачи для лагранжевых систем, теорема Гамильтона-
hi
Остроградского, формула изменения лагранжиана при замене координат и
времени, теорема Нётер}
\subsection{Переменные Гамильтона. Обобщенные импульсы}
hi
\subsection{Функция Гамильтона через функцию Лагранжа}
hi
\subsection{Функция Гамильтона через обобщенный потенциал и кин.энергию (случай
обобщенно консервативной системы) (см. 286-287 стр. Маркеева)}
hi
\subsection{Канонические уравнения Гамильтона}
hi
\subsection{Понятие первого интеграла динамической системы (не только для
гамильтоновых систем!)}
hi
\subsection{Понятие скобок Пуассона, их свойства (дистрибутивность,
антикоммутативность etc). Критерий первого интеграла гамильтоновой
системы}
hi
\subsection{Трубка прямых путей. Интегральные инварианты Пуанкаре и Пуанкаре-
Картана}
hi
\subsection{Теорема Лиувилля о сохранении фазового объема. Общая формула для
изменения фазового объема произвольной динамической системы (не только
гамильтоновой!)}
hi
\subsection{Классификация интегральных инвариантов, теорема Ли Хуачжуна}
hi
\subsection{Канонические преобразования. Производящие функции}
hi
\subsection{Замена гамильтониана при каноническом преобразовании, $(q,\,p)$ 
описание}
hi
\subsection{Свободные  преобразования, $(q,\,q^*)$ описание. Формулы
	преобразования
импульсов гамильтониана}
hi
\subsection{\emph{<<Наивная>> теория возмущений, использование $(q,\,p^*)$ описания
для задания преобразований, близких к тождественным. Метод Биркгофа, понятие
резонанса}}
hi
\subsection{Уравнение Гамильтона-Якоби. Полный интеграл уравнения
Гамильтона-Якоби}
hi
\subsection{Понятие адиабатических инвариантов динамических систем}
hi
\subsection{Переменные действие угол. Условие возможности перехода к ним,
формулы
перехода (случай одной степени свободы)}
hi
\subsection{Понятие интегрируемых гамильтоновых систем. Теорема
Лиувилля-Арнольда}
hi
\subsection{Резонансные и нерезонансные торы}
hi
\subsection{\emph{Невырожденность и изоэнергетическая невырожденность
гамильтониана}}
hi
\subsection{\emph{КАМ-теорема}}
hi
\subsection{\emph{Важные следствия КАМ-теоремы для систем с двумя степенями свободы,
обладающих свойством изоэнергетической и обычной невырожденности}}
hi
\subsection{\emph{Понятие детерминированного хаоса в динамических системах}}
hi
\subsection{\emph{Сечения Пуанкаре}}
hi
\subsection{\emph{Фрактальная размерность}}
hi
\end{document} % конец документа
