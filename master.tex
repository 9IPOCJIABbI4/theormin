
\documentclass[a4paper]{article}

% Этот шаблон документа разработан в 2014 году
% Данилом Фёдоровых (danil@fedorovykh.ru) 
% для использования в курсе 
% <<Документы и презентации в \LaTeX>>, записанном НИУ ВШЭ
% для Coursera.org: http://coursera.org/course/latex .
% Исходная версия шаблона --- 
% https://www.writelatex.com/coursera/latex/5.3

% В этом документе преамбула

%%% Работа с русским языком
\usepackage{cmap}					% поиск в PDF
\usepackage{mathtext} 				% русские буквы в формулах
\usepackage[T2A]{fontenc}			% кодировка
\usepackage[utf8x]{inputenc}			% кодировка исходного текста
\usepackage[english,russian]{babel}	% локализация и переносы
\usepackage{indentfirst}
\frenchspacing

\renewcommand{\epsilon}{\ensuremath{\varepsilon}}
\renewcommand{\phi}{\ensuremath{\varphi}}
\renewcommand{\kappa}{\ensuremath{\varkappa}}
\renewcommand{\le}{\ensuremath{\leqslant}}
\renewcommand{\leq}{\ensuremath{\leqslant}}
\renewcommand{\ge}{\ensuremath{\geqslant}}
\renewcommand{\geq}{\ensuremath{\geqslant}}
\renewcommand{\emptyset}{\varnothing}

%%% Дополнительная работа с математикой
\usepackage{amsmath,amsfonts,amssymb,amsthm,mathtools} % AMS
\usepackage{icomma} % "Умная" запятая: $0,2$ --- число, $0, 2$ --- перечисление

%% Номера формул
%\mathtoolsset{showonlyrefs=true} % Показывать номера только у тех формул, на которые есть \eqref{} в тексте.
%\usepackage{leqno} % Нумереация формул слева

%% Свои команды
\DeclareMathOperator{\sgn}{\mathop{sgn}}

%% Перенос знаков в формулах (по Львовскому)
\newcommand*{\hm}[1]{#1\nobreak\discretionary{}
{\hbox{$\mathsurround=0pt #1$}}{}}

%%% Работа с картинками
\usepackage{graphicx}  % Для вставки рисунков
\graphicspath{{figures/}}  % папки с картинками
\setlength\fboxsep{3pt} % Отступ рамки \fbox{} от рисунка
\setlength\fboxrule{1pt} % Толщина линий рамки \fbox{}
\usepackage{wrapfig} % Обтекание рисунков текстом

%%% Работа с таблицами
\usepackage{array,tabularx,tabulary,booktabs} % Дополнительная работа с таблицами
\usepackage{longtable}  % Длинные таблицы
\usepackage{multirow} % Слияние строк в таблице

%%% Теоремы
\theoremstyle{plain} % Это стиль по умолчанию, его можно не переопределять.
\newtheorem{theorem}{Теорема}
\newtheorem*{thm}{Теорема}
\newtheorem{proposition}[theorem]{Утверждение}
 
\theoremstyle{definition} % "Определение"
\newtheorem{corollary}{Следствие}[theorem]
\newtheorem*{dfn}{Определение}
\newtheorem{problem}{Задача}
\newtheorem*{problem*}{Задача}

 
\theoremstyle{remark} % "Примечание"
\newtheorem*{nonum}{Решение}

%%% Программирование
\usepackage{etoolbox} % логические операторы

%%% Страница
%\usepackage{extsizes} % Возможность сделать 14-й шрифт
%\usepackage{geometry} % Простой способ задавать поля
%	\geometry{top=25mm}
%	\geometry{bottom=35mm}
%	\geometry{left=35mm}
%	\geometry{right=20mm}
 
\usepackage{fancyhdr} % Колонтитулы
%	\pagestyle{fancy}
 %	\renewcommand{\headrulewidth}{0pt}  % Толщина линейки, отчеркивающей верхний колонтитул
	%\lfoot{Нижний левый}
	%\rfoot{Нижний правый}
	%\rhead{Верхний правый}
	%\chead{Верхний в центре}
	%\lhead{Верхний левый}
	%\cfoot{Нижний в центре} % По умолчанию здесь номер страницы

\usepackage{setspace} % Интерлиньяж
%\onehalfspacing % Интерлиньяж 1.5
%\doublespacing % Интерлиньяж 2
%\singlespacing % Интерлиньяж 1

\usepackage{lastpage} % Узнать, сколько всего страниц в документе.

\usepackage{soul} % Модификаторы начертания

\usepackage{hyperref}
\usepackage[usenames,dvipsnames,svgnames,table,rgb,pdftex,dvipsnames]{xcolor}
\hypersetup{				% Гиперссылки
    unicode=true,           % русские буквы в раздела PDF
    pdftitle={Заголовок},   % Заголовок
    pdfauthor={Автор},      % Автор
    pdfsubject={Тема},      % Тема
    pdfcreator={Создатель}, % Создатель
    pdfproducer={Производитель}, % Производитель
    pdfkeywords={keyword1} {key2} {key3}, % Ключевые слова
    colorlinks=true,       	% false: ссылки в рамках; true: цветные ссылки
    linkcolor=red,          % внутренние ссылки
    citecolor=black,        % на библиографию
    filecolor=magenta,      % на файлы
    urlcolor=cyan           % на URL
}

\usepackage{csquotes} % Еще инструменты для ссылок

%\usepackage[style=apa,maxcitenames=2,backend=biber,sorting=nty]{biblatex}

\usepackage{multicol} % Несколько колонок

\usepackage{tikz} % Работа с графикой
\usepackage{pgfplots}
\usepackage{pgfplotstable}
\usepackage{coloremoji}
\usepackage{floatrow}
\usepackage{subcaption}
\newcommand*{\N}{\mathbb{N}}
\newcommand*{\R}{\mathbb{R}}
\newcommand*{\K}{\mathbb{K}}
\newcommand*{\V}{\mathcal{V}}
\newcommand*{\A}{\mathcal{A}}
\newcommand*{\ii}{\mathbf{1}}
\newcommand*{\oo}{\mathbf{0}}
\newcommand*{\ba}{\mathbf{a}}
\newcommand*{\bb}{\mathbf{b}}
\newcommand*{\Q}{\mathbb{Q}}
\graphicspath{{images/}}
%\usepackage{breqn}

\renewcommand\thesubfigure{\asbuk{subfigure}}
%\addbibresource{master.bib}

\usepackage{import}
\usepackage{pdfpages}
\usepackage{transparent}
\usepackage{xcolor}
\usepackage{xifthen}

%\newcommand{\incfig}[1]{%
%    \def\svgwidth{\columnwidth}
%    \import{./figures/}{#1.pdf_tex}
%}


\newcommand{\incfig}[2][1]{%
    \def\svgwidth{#1\columnwidth}
    \import{./figures/}{#2.pdf_tex}
}
\usepackage{titlesec}
\titleformat{\section}{\normalfont\Large\bfseries}{}{0pt}{}


\pdfsuppresswarningpagegroup=1
\pgfplotsset{compat=1.16}
\renewcommand{\thesection}{}
\renewcommand{\thesubsection}{\arabic{subsection}}

\usepackage{marginnote}
\usepackage{xargs}
\usepackage[colorinlistoftodos,prependcaption,shadow, textwidth=3cm]{todonotes}
\newcommandx{\improvement}[2][1=]{\todo[disable,linecolor=OliveGreen,backgroundcolor=OliveGreen!25,bordercolor=OliveGreen,#1]{#2}}
\newcommandx{\unsure}[2][1=]{\todo[disable,linecolor=YellowOrange,backgroundcolor=YellowOrange!25,bordercolor=YellowOrange,#1]{#2}}
\newcommandx{\didntdone}[2][1=]{\todo[disable,linecolor=red,backgroundcolor=red!25,bordercolor=red,#1]{#2}}
\newcommandx{\impr}[2][1=]{\todo[linecolor=OliveGreen,backgroundcolor=OliveGreen!25,bordercolor=OliveGreen,#1]{#2}}
%%\setcounter{secnumdepth}{0}
%%% Local Variables:
%%% mode: latex
%%% TeX-master: "master"
%%% End:


\title{Теорминимум
по теормеху}
\author{Драчов Ярослав\\ Факультет общей и прикладной физики, МФТИ}


\begin{document} % конец преамбулы, начало документа
%\input{titlepage}
%\pdfsuppresswarningpagegroup=1a
\maketitle
\listoftodos[Исправления]
\tableofcontents

\section{Первое задание}
%\addcontentsline{toc}{section}{Первое задание}
\subsection{Определение положения равновесия}
\begin{dfn}
	%Некоторое положение системы тогда и только тогда является её
	%\emph{положением
	%равновесия}, когда в этом положении все обобщённые силы равны нулю:
	%\[
	%	Q_i=-\frac{\partial \Pi}{\partial q_i} =0, \quad (i=1,\,2,\ldots
	%	,\,n),
	%\] 
	%где $\Pi$ --- потенциальная энергия системы, которая в случае
	%консервативной системы явно от времени не зависит.
	\improvement{Положение равновесия ред.}Положение системы материальных точек,
	определяемое в некоторой системе
	отсчёта обобщёнными координатами $q_j=q_j^n\,\left( j=\overline{1,\,n} \right) $,
	называется \emph{положением равновесия} для наблюдателя, связанного
	с этой системой отсчёта, если система материальных точек, будучи
	приведена в это положение с нулевыми скоростями $\dot{q}_j^0\,
	\left( j= \overline{1,\,n} \right) $, остаётся в нём сколь угодно долго.
\end{dfn}
\subsection{Устойчивое положение равновесия}
Пусть \improvement{Добавлена система дифуров, описывающих движение}
	поставлена задача Коши для нахождения движения системы:
	\[
		\mathbf{f}\left( \mathbf{q},\,\dot{\mathbf{q}},
		\ddot{\mathbf{q}},\,t \right) =\mathbf{0}, \qquad
		\mathbf{q}(t_0)=\mathbf{q}^0,\qquad
		\dot{\mathbf{q}}(t_0)=\dot{\mathbf{q}}^0
	.\] 
\begin{dfn}
	Положение равновесия $\mathbf{q}=\mathbf{0}$ называется  \emph{устойчивым},
	если для любого $\epsilon >0$ существует такое $\delta=\delta(\epsilon)$,
	что для всех $t>t_0$ выполняются неравенства
	\[
		|q_i(t)|<\epsilon,\qquad \left|\dot{q}_i(t)\right|<\epsilon
		\qquad(i=1,\,2,\ldots
		,\,n)
	\] 
	где $q_i(t)$ --- решения поставленной задачи Коши,
	при условии, что в начальный момент $t=t_0$ 
	\[
	|q_i^0|<\delta,\qquad \left|\dot{q}_i^0\right|<\delta
	.\] 
\end{dfn}
\subsection{Теорема Лагранжа-Дирихле}
\begin{thm}[Лагранжа-Дирихле]
	Если в положении равновесия консервативной системы потенциальная 
	энергия имеет строгий  локальный минимум,  то это положение равновесия
	устойчиво.
\end{thm}
\subsection{Первая теорема Ляпунова о неустойчивости}
\begin{thm}[Ляпунова, 1-я]
Если потенцмальная энергия консервативной системы в положении равновесия не
имеет минимума и это узнаётся уже по членам второго порядка в разложении
функции $\Pi$ в ряд в окрестности положения равновесия без необходимости
рассматривания членов высших порядков, то положение равновесия неустойчиво.
\end{thm}
\subsection{Вторая теорема Ляпунова о неустойчивости}
\begin{thm}[Ляпунова, 2-я]
	Если в положении равновесия потенциальная энергия имеет максимум и
	это узнаётся по членам наименее высокого порядка, которые действительно
	присутствуют в разложении этой функции в ряд в окрестности положения
	равновесия, то это положение равновесия неустойчиво.
\end{thm}
\subsection{Нормальные координаты}
\begin{dfn}
	Обобщённые координаты $\theta_j$, в которых кинетическая и потенциальная
	энергия системы имеют вид
	\[
		T=\frac{1}{2} \sum_{j=1}^{n} \dot{\theta}_j^2, \qquad
		\Pi=\frac{1}{2} \sum_{j=1}^{n} \lambda_j \theta_j^2,
	\]
	называются \emph{нормальными}.
\end{dfn}
Привести систему
к нормальным координатам можно в случае, если кинетическая
энергия в первоначальных переменных --- положительно-определённая квадратичная
форма скоростей, а потенциальная энергия произвольная квадратичная форма
координат.\improvement{Добавлено условие приводимости к нормальной форме}

В нормальных координатах лианеризованные уравнения движения в окрестности
положения равновесия имеют вид $n$ не связанных друг с другом уравнений
второго порядка
\[
	\ddot{\theta}_j+\lambda_j \theta_j =0, \quad (j =1,\,2,\ldots,\,n)
.\] 
\subsection{Асимптотически устойчивое положение равновесия}
	Пусть \improvement{Добавлена система дифуров, в соответствии с этим
	отредактирована формулировка}
	поставлена задача Коши для нахождения движения системы:
	\[
		\mathbf{f}\left( \mathbf{q},\,\dot{\mathbf{q}},
		\ddot{\mathbf{q}},\,t \right) =\mathbf{0}, \qquad
		\mathbf{q}(t_0)=\mathbf{q}^0,\qquad
		\dot{\mathbf{q}}(t_0)=\dot{\mathbf{q}}^0
	.\] 
\begin{dfn}
	Положение равновесия $\mathbf{q}=\mathbf{0}$ называется \emph{асимптотически
	устойчивым},
	если оно устойчиво и, если, кроме того, существует такая 
	$\delta$-окрестность точки $\mathbf{q}=\mathbf{0}$, $\dot{\mathbf{q}}=0$,
	что для всех $|q_j^0|<\delta$, $|\dot{q}_j^0|< \delta$ выполняются
	условия
	\[
		\lim_{t \to \infty} q_j (t)=0, \qquad \lim_{t \to \infty} 
		\dot{q}_j(t)=0,\qquad (j=1,\ldots,\,n)
	,\]
	где $q_j(t)$ --- решения поставленной задачи Коши.
\end{dfn}
\subsection{Теорема Ляпунова о лианеризованных системах}
\begin{thm}[Ляпунова о  лианеризованных системах]
	Если все корни характеристического уравнения
\[
	\det \|\mathbf{A}\lambda^2+\mathbf{B}^* \lambda +(\mathbf{C}+
	\mathbf{C}^*)\|=0
\]
системы дифференциальных уравнений линейного приближения
\[
	\mathbf{A}\ddot{\mathbf{q}}+\mathbf{B}^* \dot{\mathbf{q}}+
	(\mathbf{C}+\mathbf{C}^*)\mathbf{q}=0
\]
имеют отрицательные действительные части, то положение равновесия  $q=0$
исходной \emph{стационарной}\improvement{Добавлено упоминание стационарности
системы, звёздочки не убирал, они общность не умаляют вроде, а показывают что
в  системе связано с обобщёнными силами}
системы, описываемой уравнениями
\[
	\frac{d}{dt} \frac{\partial T}{\partial \dot{q}_j} -\frac{\partial T}{\partial q_j} =
	Q_j,\qquad Q_j=-\frac{\partial V}{\partial q_j} +Q_j^* \quad (j=1,\ldots,
	\,n)
,\] 
асимтотически устойчиво. Если хотя бы один корень характеристического уравнения
имеет действительную часть, то положение равновесия, определяемое системой,
неустойчиво.
\end{thm}
\subsection{Критерии Рауса-Гурвица и Льенара-Шипара устойчивости многочлена}
\begin{dfn}
	 Назовём \emph{матрицей Гурвица} квадратную  матрицу $m$-го порядка
	 \[
		 \begin{pmatrix} a_1 & a_3 & a_5 & \ldots & 0\\
		 a_0 & a_2 & a_4 & \ldots & 0 \\
	 0 & a_1 & a_3 & \ldots & 0\\
 0 & a_0 & a_2 & \ldots & 0 \\
  & & & \ddots & \vdots\\
  & & & & a_m
 \end{pmatrix} 
	  .\] 
\end{dfn}
Составим главные миноры  матрицы Гурвица (\emph{определители Гурвица})
\[
	\Delta_1=a_1, \; \Delta_2 = \begin{pmatrix} a_1 & a_3 \\ a_0 & a_2 \end{pmatrix},
	\; \Delta_3= \begin{pmatrix} a_1 & a_3 & a_5 \\ a_0 & a_2 & a_4 \\
	0 & a_1 & a_3\end{pmatrix} ,\ldots,\; \Delta_m =a_m \Delta_{m-1}
.\] 
\begin{thm}[Критерий Рауса-Гурвица]
	Для того чтобы все корни уравнения
	\[
	a_0 \lambda^m+a_1 \lambda^{m-1}+\ldots+a_{m-1}\lambda+a_m=0
	\]
	с вещественнными коэффициентами и положительным старшим  коэффициентом
	$a_0$ имели отрицательные вещественные   части, необходимо и достаточно,
	чтобы выполнялись неравенства
	\[
	\Delta_1>0,\quad\Delta_2>0,\quad \ldots, \quad \Delta_m >0
	.\] 
\end{thm}
\begin{thm}[Критерий Льенара-Шипара]
	\improvement{Добавлен критерий Льенара-Шипара} Для того чтобы многочлен
	$f(\lambda)\equiv a_0 \lambda^n+a_1 
	\lambda^{n-1}+\ldots+a_{n-1}\lambda+a_n$ при $a_0>0$ имел все корни
	с отрицательными вещественными частями, необходимо и достаточно, чтобы
	\begin{enumerate}
		\item все коэффициенты многочлена $f(\lambda)$ были
			положительны
			 \[
			a_1>0,\qquad a_2>0,\qquad \ldots, \qquad a_n>0;
			\]
		\item %имели место детерминантные неравенства
		%	 \[
		%	\Delta_{n-1}>0, \qquad \Delta_{n-3}>0,\qquad \ldots
		%	\]
			в последовательности определителей Гурвица все
			определители с чётными индексами (либо все определители
			с нечётными индексами) были строго положительными.
			\impr{Исп.}
	\end{enumerate}
	%(здесь, как и ранее, $\Delta_k$ обозначает определитель Гурвица $k$-го
	%порядка).
\end{thm}
\subsection{Теорема Ляпунова об\\
	устойчивости/асимптотической устойчивости\\
(функция Ляпунова)}\impr{Формулировки по Амелькину}
%\begin{thm}[Ляпунова, об устойчивости движения]
%	Если дифференциальные уравнения возмущённого движения таковы, что
%	существует знакоопределённая функция $V$, производная которой $\dot{V}$ 
%	в силу этих уравнений является  или  знакопостоянной функцией
%	противоположного знака с $V$, или тождественно равнойнулю,\unsure{
%		здесь упомянуто про тождественное равенство нулю,
%		полуопределённость всё же больше для форм подходит, я пока что
%	решил не исправлять}
%		то
%	невозмущённое движение устойчиво.
%\end{thm}
\begin{thm}[Ляпунова, об устойчивости движения]
	Если существует функция $V(\mathbf{x})$, определённо положительная
	в окрестности положения равновесия $\mathbf{x}=\mathbf{0}$ (имеет
	строгий минимум в точке $\mathbf{x}=\mathbf{0}$), а её производная
	по времени, вычисленная в силу уравнений
	\[
				\dot{\mathbf{x}}=\mathbf{f}(\mathbf{x}) \tag{*}
				\label{eq:hi2}
			,\]
	знакоотрицательна, т.\:е.
\[
			\left. \dot{V}  \right|_{(\ref{eq:hi2})}=
			\left.\frac{\partial V}{\partial \mathbf{x}} \dot{\mathbf{x}}
			\right|_{(\ref{eq:hi2})}=\frac{\partial V}{\partial  \mathbf{x}}
			\mathbf{f}\le 0
		,\]
	то $\mathbf{x}=\mathbf{0}$ --- устойчивое положение равновесия.
\end{thm}
\begin{thm}[Ляпунова, об асимптотической устойчивости]
	Если существует функция $V(\mathbf{x})$, определённо положительная
	в окрестности положения равновесия $\mathbf{x}=\mathbf{0}$ (имеет
	строгий минимум в точке $\mathbf{x}=\mathbf{0}$), а её производная
	по времени, вычисленная в силу уравнений (\ref{eq:hi2})
	определённо отрицательна, т.\:е.
\[
			\left. \dot{V}  \right|_{(\ref{eq:hi2})}=
			\left.\frac{\partial V}{\partial \mathbf{x}} \dot{\mathbf{x}}
			\right|_{(\ref{eq:hi2})}=\frac{\partial V}{\partial  \mathbf{x}}
			\mathbf{f}< 0
		,\]
	то $\mathbf{x}=\mathbf{0}$ --- устойчивое положение равновесия.
\end{thm}
%\begin{thm}[Ляпунова, об асимптотической устойчивости]
%	Если дифференциальные уравнения возмущённого движения таковы, что
%	существует знакоопределённая функция $V(x_1,\,x_2,\ldots,\,x_m)$,
%	производная которой $\dot{V}$  в силу этих уравнений  есть
%	знакоопределённая функция противоположного знака с $V$, то
%	невозмущённое движение асимптотически устойчиво.
%\end{thm}
\subsection{Теорема Барбашина-Красовского}\improvement{Формулировка по Амелькину}
%\begin{thm}[Барбашина-Красовского]
%	Если для дифференциальных уравнений возмущённого движения можно найти
%	положительно определённую функцию $V(y)$, производная которой,
%	вычисленная в силу этих уравнений $\dot{V}(y)<0$ вне $K$ и
%	$\dot{V}(y)=0$ на $K$, где $K$  --- многообразие точек, содержащее
%	единственное решение $y(t) \equiv 0$, то невозмущённое движение
%	асимптотически устойчиво, если функция $V(y)$ отрицательно определена,
%	то невозмущённое движение асимптотически неустойчиво.
%\end{thm}
\begin{thm}[Барбашина-Красовского]
	Пусть существует функция $V(\mathbf{x})$, для которой в некоторой окрестности
	положения равносвесия $\mathbf{x}=\mathbf{0}$ выполняются условия:
	\begin{enumerate}
		\item Её производная по времени, вычисленная в силу уравнений
			\[
				\dot{\mathbf{x}}=\mathbf{f}(\mathbf{x}) \tag{*}
				\label{eq:hi}
			,\]
		знакоотрицательна, т.\:е.
		\[
			\left. \dot{V}  \right|_{(\ref{eq:hi})}=
			\left.\frac{\partial V}{\partial \mathbf{x}} \dot{\mathbf{x}}
			\right|_{(\ref{eq:hi})}=\frac{\partial V}{\partial  \mathbf{x}}
			\mathbf{f}\le 0
		.\]
		\item Множество $X^0$ точек  $\mathbf{x}$, в которых
			производная $\dot{V}$ равна нулю:
 \[
	 X^0:\: \left.\dot{V}(\mathbf{x})\right|_{(\ref{eq:hi})}=
		 \frac{\partial V}{\partial \mathbf{x}}\mathbf{f}=0 
 \]
 кроме $\mathbf{x}(t)=\mathbf{0}$ не содержит других целых траекторий системы
 (\ref{eq:hi}). ($\mathbf{x}(t)=\mathbf{0}$ --- единственное решение, полностью
 лежащее на множестве $X^0$).
	\end{enumerate}
	Тогда
	\begin{itemize}
		\item[а)] Если функция $V(\mathbf{x})$ имеет строгий минимум в
			точке $\mathbf{x}=\mathbf{0}$, то это положение
			асимптотически устойчиво.
		\item[б)] Если в точке $\mathbf{x}=\mathbf{0}$ функция
				$V(\mathbf{x})$ не имеет минимума (включая
				нестрогий)\impr{Исп.}, то это положение
				неустойчиво.
	\end{itemize}
\end{thm}

\subsection{Теорема Четаева о неустойчивости}\impr{Формулировка по Амелькину}
%\begin{thm}[Четаева, о неустойчивости]
%	Если дифференциальные уравнения возмущённого движения таковы, что
%	существует функция\\$V(x_1,\,x_2,\ldots,\,x_m)$
%	такая, что в сколь угодно
%	малой  окрестности
%	 \[
%		 |x_i|<h \quad (i=1,\,2,\ldots,\,m)
%	\] 
%	существует область $V>0$ и во всех точках области $V>0$ производная
%	$\dot{V}$ в силу этих уравнений принимает положительные значения, то
%	невозмущённое движение неустойчиво.
%\end{thm}
\begin{thm}[Четаева, о неустойчивости]
	Пусть существует такая функция $V(\mathbf{x})$, что в любой достаточно
	малой окрестности точки $\mathbf{x}=\mathbf{0}$ существует область
	$V<0$. Если во всех точках области $V<0$ производная  по времени,
	вычисленная в силу уравнений
			\[
				\dot{\mathbf{x}}=\mathbf{f}(\mathbf{x}) \tag{*}
				\label{eq:hi_3}
			,\]
	определённо отрицательна, т.\:е.
\[
			\left. \dot{V}  \right|_{(\ref{eq:hi2})}=
			\left.\frac{\partial V}{\partial \mathbf{x}} \dot{\mathbf{x}}
			\right|_{(\ref{eq:hi2})}=\frac{\partial V}{\partial  \mathbf{x}}
			\mathbf{f}< 0
		,\]
	то положение равновесия $\mathbf{x}=\mathbf{0}$ неустойчиво.
\end{thm}

\subsection{Понятие о бифуркации положений равновесия}
Дифференциальные уравнения динамических систем часто зависят не только от
фазовых переменных, но и от некоторых параметров.
%В  этом случае уравнения
%имеют вид
%\[
%	\dot{\mathbf{x}} = \mathbf{f}(\mathbf{x},\,\boldsymbol{\alpha}),\quad
%	\mathbf{x} \in \R^n, \quad \boldsymbol{\alpha} \in \R^m
%,\] 
%где $\mathbf{x}$ --- фазовый вектор, $\boldsymbol{\alpha}$ --- набор параметров.
Иногда изменение параметров приводит к качественным перестройкам  структуры
фазовых траекторий системы --- изменениям количества положений равновесия,
характера их устойчивости, кардинальным трансформациям траекторий и т.\;п.
В окрестности определённых значений параметров перестройки происходят при
сколь угодно малом изменении параметров и называются \emph{бифуркациями}.
\subsection{Бифуркация Андронова-Хопфа}\improvement{Новые определения}
\begin{dfn}
	\emph{Предельным циклом} векторного поля на фазовой плоскости или,
	более обобщённо, на каком-либо думерном многообразии называется
	замкнутая (периодическая) траектория этого векторного поля, в
	окрестности которой нет других периодических траекторий.
\end{dfn}
\begin{dfn}
	\emph{Бифуркация Андронова-Хопфа} --- локальная бифуркация векторного
	поля на плоскости, в ходе которой особая точка-фокус теряет
	устойчивость при переходе пары её комплексно-сопряжённых собственных
	значений через мнимую ось. При этом либо из особой точки рождается
	небольшой устойчивый предельный цикл, либо, наоборот, небольшой
	нейстойчивый предельный цикл в момент бифуркации схлопывается в эту
	точку.
\end{dfn}
%\emph{Бифуркацией Андронова-Хопфа} называется бифуркация рождения цикла.
\subsection{Метод Пуанкаре\improvement{Исп.} приведения к нормальной форме.
Понятие резонанса}
\begin{dfn}
	\emph{Нормальной формой} системы
	\[
		\dot{\mathbf{x}}=\mathbf{f}(\mathbf{x}),\quad
		\mathbf{x}=[x_1,\ldots,\,x_n]^T, \quad 
		\mathbf{f}=[f_1,\ldots,\,f_n]^T
	\]
	называется форма содержащая лишь линейные и резонансные слагаемые.
\end{dfn}
При помощи разложения в ряд Тейлора функции   $\mathbf{f}$ в окрестности
положения равновесия выделим в данной системе линейные слагаемые:
\begin{gather*}
	\dot{\mathbf{x}}= \mathbf{Ax}+\boldsymbol{g}(\mathbf{x}),\quad
	\boldsymbol{g}=[g_1,\ldots,\,g_n]^T,\\ \quad g_i(\mathbf{x})=
	\sum_{k_1,\ldots,\,k_n}^{} g_{k_1,\ldots,\,k_n}^i x_1^{k_1}\ldots
	x_n ^{k_n},\quad k_1+\ldots+k_n \ge 2
.\end{gather*}
Здесь $\mathbf{A}$ --- матрица с постоянными коэффициентами размера $n \times n$,
$g_{k_1,\ldots,\,k_n}$ --- постоянные коэффициенты в полиномах $\boldsymbol{g}$.
\begin{dfn}
	\emph{Нормализующим преобразованием} (вплоть до степени $k$) 
	называется последовательность преобразований
	\begin{gather*}
		\tilde{\mathbf{x}}=\mathbf{y}+ \mathbf{p}(\mathbf{y}),\quad
		\mathbf{p}(\mathbf{y})=[p_1,\ldots,\,p_n]^T,\\
		p_i(\mathbf{y})= \sum_{k_1,\ldots,\,k_n}^{} p^i_{k_1,\ldots,\,k_n}
		y_1^{k_1}\ldots y_n^{k_n},\quad k_1+\ldots+k_n=k
	\end{gather*}
	с полиномиальными коэффициентами
	\[
	p_{k_1,\ldots,\,k_n}^i=\frac{g_{k_1,\ldots,\,k_n}^i}{k_1 \lambda_1+
	\ldots+k_n \lambda_n-\lambda_i}
	,\] 
	приводящее систему к её нормальной форме (вплоть до слагаемых степени
	$k$).
\end{dfn}
\begin{dfn}
	Нелинейные слагаемые системы, показатели $k_1,\ldots,\,k_n$ которых
	таковы, что выполняется $\lambda_i=k_1 \lambda_1+\ldots k_n \lambda_n$,
	называются \emph{резонансными}.
\end{dfn}
В соответствии с описанным выше алгоритмом все нерезонансные слагаемые могут
быть исключены из правой части системы при помощи последовательно применяемых
полиномиальных замен. В то же время резонансные слагаемые не могут быть ни
исключены, ни каким-либо образом ими преобразованы.
\subsection{Вынужденные колебания под действием периодической силы. Частотная
характеристика, амплитудно-фазовая характеристика. Необходимые и достаточные
условия возникновения резонанса в таких системах.}
\begin{dfn}
	Колебания, которые возникают благодаря наличию вынуждающей силы,
	зависящей явно от времени, и к которым в пределе стремится суммарное
	движение, назывют \emph{вынужденными колебаниями}.
\end{dfn}
Представим уравнения линейного приближения стационарной системы в виде
\[
	\sum_{k=1}^{n} (a_{jk}\ddot{q}_k+b_{jk}\dot{q}_k+c_{jk}q_k)=Q_j (t)
	\quad (j=1,\ldots,\,n),
\] 
где $Q_j (t)$ --- зависящие явно от времени части обобщённых сил.
Предполагается, что в процессе движения они остаются малыми по модулю и не
выводят систему из малой окрестности положения равновесия.

В связи с тем, что данная система уравнений является линейной, а для линейных
систем имеет место принцип суперпозиции, можно рассмотреть движение системы
под действием какой-либо одной силы из $Q_j(t)$ $(j=1,\ldots,\,n)$, предположив,
что все остальные равны нулю. Определив порознь движения, возникающие под
действием каждой из таких обобщённых сил, их следует затем сложить.

Учитывая это обстоятельство, положим
\[
	Q_2(t)=Q_3(t)=\ldots=Q_n(t)=0,
\]
т.\:е. будем считать, что отлична от нуля только обобщённая сила $Q_1(t)$,
относящаяся к первой обобщённой координате, а все остальные обобщённые силы
такого рода равны нулю.

Предположим\improvement{Добавлено упоминание гармоничности}, что обобщённая
сила $Q_1$ является гармонической функцией от $t$,
\[
	Q_1(t)=A \sin \Omega t
.\]
Здесь $A$ --- амплитуда, а $\Omega$ --- частота внешней вынуждающей силы $Q_1$.
Если же в системе присутствуют обобщённые периодические силы $Q_j$, которые
гармоническими функциями от  $t$ не являются, то их мы можем представить в
виде ряда Фурье гармонических функций и далее работать с этим рядом.
 
Введём обозначения
\begin{gather*}
	d_{jk}(i\Omega)=a_{jk}(i\Omega)^2+b_{jk}(i\Omega)+c_{jk},\\
	\Delta=\det \|d_{jk}(i\Omega)\|=\begin{pmatrix} d_{11}(i\Omega) &
	\cdots & d_{1n}(i\Omega) \\ \vdots & & \vdots \\ d_{n1}(i\Omega) &
\cdots & d_{nn}(i\Omega)\end{pmatrix} ,\\
W_{1k}(i\Omega)=\frac{\Delta_{1k}}{\Delta},
\end{gather*}
где  $\Delta_{1k}$ --- алгебраическое дополнение расположенного в первой строке
и $k$-ом столбце  элемента определителя $\Delta$.
Тогда при поиске частного решения неоднородной системы дифференциальных уравнений
в виде $\tilde{q}_f=B e ^{i\Omega t}$ и, выделяя в последствии мнимую часть,
получим
\[
	q_k=A |W_{1k}(i\Omega)|\sin[\Omega t +\arg W_{1k}(i\Omega)]
	\quad (k=1,\ldots,\,n)
.\]
\begin{dfn}
	Введённая выше функция $W_{1k}(i\Omega)$ называется \emph{частотной
	характеристикой системы}, или, как говорят иногда, её
	\emph{амлитудно-фазовой
характеристикой}.
\end{dfn}
\begin{dfn}
	 Если отдельно рассмотреть изменение модуля и аргумента вектора
	$W(i\Omega)$ в зависимости от $\Omega$, то получатся характеристики,
	которые называются соответственно \emph{амплитудной} и \emph{фазовой}
	характеристиками системы.
\end{dfn}
\begin{dfn}
	Если амплитудная характеристика системы при некотором значении $\Omega=
	\Omega^*$ 
	имеет отчётливо выраженный пик, то при одной и той же амплитуде
	внешней силы $Q_1$ амплитуда отклика резко возрастает, когда частота
	внешней силы приближается к значению $\Omega=\Omega^*$. Это явление
	называют \emph{резонансом (в геометрическом смысле)}.
	\improvement{Необходимые и достаточные
	условия резонанса}
\end{dfn}
\begin{dfn}
	\emph{Резонансом (в смысле существенного изменения частного решения)}
	будем называть случай, когда система не имеет частного решения
	представленного суперпозицией гармонических функций с частотой
	внешней вынуждающей силы.
\end{dfn}
Пусть
\[
	A=\|a_{jk}\|,\qquad B=\|b_{jk}\|,\qquad C=\|c_{jk}\|,
	\qquad \mathbf{Q}(t)=\mathbf{a}\sin \Omega t
.\]
Тогда неразрешимость матричного уравения 
\[
	(-A \omega^2+ i B \omega+C)\mathbf{b}=\mathbf{a}
\]
для любого вектора обобщённых сил $\mathbf{Q}(t)$, действующих в системе 
будет \emph{необходимым и достаточным условием резонанса (в смысле
существенного изменения частного решения)}.

В данном случае условие
\[
	\det (-A \omega^2 +i B\omega+C)=0
\] 
будет  \emph{необходимым}.
\section{Второе задание}
%\addcontentsline{toc}{section}{Второе задание}
%\setcounter{subsection}{0}
\subsection{Понятие краевой задачи для лагранжевых систем, теорема Гамильтона-
Остроградского, формула изменения лагранжиана при замене координат и
времени, теорема Нётер}\improvement{Добавлены определение окольного путя,
действия по Гамильтону, поставлена вариационная задача}
\begin{dfn}
	\emph{Прямым путём} системыназывают путь этой системы из точки $A$ в
	точку  $B$ в $(n+1)$-мерном расширенном координатном пространстве
	$q_1,\ldots,\,q_n,t$, удовлетворяющий соответствующим уравнениям
	Лагранжа.
\end{dfn}
\begin{dfn}
	\emph{Окольным путём} системы называют путь этой системы из точки $A$ в
	точку  $B$ в $(n+1)$-мерном расширенном координатном пространстве
	$q_1,\ldots,\,q_n,t$, не являющийся прямым.
\end{dfn}
Рассмотрим произвольную голономную систему с независимыми координатами
$q_1,\ldots,\,q_n$ и функцией Лагранжа $L\left(t,\,q_i,\,\dot{q}_i\right)$.
\begin{dfn}
	Интеграл
	\[
	W=\int\limits_{t_0}^{t_1} L dt 
	\] 
	называется \emph{действием (по Гамильтону)} за промежуток времени
	$(t_0,\,t_1)$.
\end{dfn}
При построении прямого пути системы решается вариационная задача о нахождении
экстремали функционала
\[
	\int\limits_{t_0}^{t_1} L\left( t,\,q_i,\,\dot{q}_i \right) dt
\] с заданными граничными условиями $q_i(t_0)=q_i^0,\,
q_i(t_1)=q_i^1,\, \left( i=\overline{1,\,n} \right) $.\impr{Исп.}
\begin{thm}[Гамильтона-Остроградского]
	Прямой путь является экстремалью рассматриваемой вариационной задачи
	--- на прямом пути действие по Гамильтону достигает стационарного
	значения.
\end{thm}
Рассмотрим преобразования
\[
	q_j=\phi_j(q^*,\,t^*),\quad t=\psi(q^*,\,t^*), \quad j=1,\ldots,\,n,
\] 
где $q^*$ и $t^*$ ---  <<новые>> координаты и время, $q$ и $t$ --- <<старые>>
координаты и время, а $\phi_j$ и $\psi$ --- достаточно гладкие функции.
Предположим, что данные преобразования разрешимы относительно переменных
$q^*$ и  $t^*$.

\emph{Формула изменения лагранжиана при замене координат и времени} тогда
выглядит следующим образом
\[
	L^*(q^*,\,dq^* /dt^*,\,t^*)=L(\phi,\,d\phi /d\psi,\,\psi)(d\psi /dt^*)
,\]
где, в свою очередь,
\[
\frac{d\phi_j}{d\psi}=\dfrac{\displaystyle\sum\limits_{k}
	\dfrac{\partial \phi_j}{\partial q_k^*} 
\dfrac{dq^*_k}{dt^*}+\dfrac{\partial \phi_j}{\partial t^*} }{\displaystyle\sum\limits_{k}
\dfrac{\partial \psi}{\partial q_k^*} 
\dfrac{dq^*_k}{dt^*}+\dfrac{\partial \psi}{\partial t^*} }
,\]
а $d \psi /dt^*$ совпадает со знаменателем данной дроби.
\begin{thm}[Нётер]
	Пусть задана система движущихся в потенциальном поле материальных
	точек, имеющая лагранжиан $L(q,\, dg /dt,\, t)$, и пусть существует
	однопараметрическое семейство преобразований
	\[
		q_j^*=\phi_j(q,\,t,\,\alpha) \quad (j=1,\ldots,\, n),
		\qquad t^*=\psi(q,\,t,\,\alpha),
	\]
	($\phi,\,\psi$ --- достаточно гладкие, обратимые функции)\improvement{
	Упомянута гладкость}
	удовлетворяющее условиям
	\begin{itemize}
		\item тождественности при $\alpha=0$, т.\:е.
			\[
				\phi_j(q,\,t,\,0)=q_j \quad(j=1,\ldots,\,n),
				\qquad \psi(q,\,t,\,0)=t;
			\]
		\item существования обратного преобразования:
			\[
				q_j= \tilde{\phi}_j(q^*,\,t^*,\,\alpha) \quad
				(j=1,\ldots,\,n), \qquad
				t=\tilde{\psi}(q^*,\,t^*,\,\alpha)
			.\] 
	\end{itemize}
	Пусть, далее, лагранжиан $L$ инвариантен по отношению к таким
	преобразованиям,  т.\:е. <<новый>> лагранжиан  $L^*$ (вычисленный по
	формуле выше) не зависит от $\alpha$ и как функция $q^*,\,dq^* /dt^*,\,
	t^*$ имеет совершенно такой же вид, как и <<старый>> лагранжиан
	$L$ как функция $q,\,dq /dt,\,t$. Тогда существует функция $\Phi(q,\,p,\,t)$,
	которая не изменяется во время движения этой системы, т.\:е.
	является первым интегралом движения. Эта функция имеет вид
	\[
		\Phi(q,\,p,\,t)= \sum p_j \left.\left(\frac{\partial \phi_j}{\partial \alpha}
			\right)\right|_{\alpha=0}-H\left.\left( \frac{\partial
			\psi}{\partial \alpha}  \right) \right|_{\alpha=0}
	,\] 
	где $H$ --- гамильтониан рассматриваемой системы.
\end{thm}
\subsection{Переменные Гамильтона. Обобщенные импульсы}
В предположении, что\improvement{Добавлено условие на гессиан}
\[
	\det \left\|\frac{\partial ^2 L}{\partial \dot{q}_j \partial \dot{q}_k} \right\|^n
	_{j,\,k=1}\neq 0
,\] 
состояние системы можно задавать при помощи параметров $q_i,\,p_i,\,t$, где
 $p_i$ --- \emph{обобщённые импульсы}, определяемые равенствами
 \[
	 p_i=\frac{\partial L}{\partial \dot{q}_i} \quad (i=1,\ldots,\,n)
 .\] 
\begin{dfn}
	Переменные $q_i,\,p_i,\,t$ называют \emph{переменными Гамильтона}.
\end{dfn}
\subsection{Функция Гамильтона через функцию Лагранжа}
В предположении, что\improvement{Добавлено условие на гессиан}
\[
	\det \left\|\frac{\partial ^2 L}{\partial \dot{q}_j \partial \dot{q}_k} \right\|^n
	_{j,\,k=1}\neq 0
,\]
функцию Гамильтона через функцию Лагранжа можно получить при помощи 
\emph{преобразования Лежандра} функции $L(q_i,\,\dot{q}_i,\,t)$ по переменным
$\dot{q}_i,\;(i=1,\,2,\ldots,\,n)$
\[
	H(q_i,\,p_i,\,t)= \sum_{i=1}^{n} p_i \dot{q}_i -L(q_j,\,\dot{q}_j,t)
,\]
где величины $\dot{q}_i$ выражены через $q_j,\,p_j,\,t$.
\subsection{Функция Гамильтона через обобщенный потенциал и кин.энергию (случай
обобщенно консервативной системы) (см. 286-287 стр. Маркеева)}
Система называется \emph{обобщённо консервативной}, если её функция Гамильтона
не зависит явно от времени. В этом случае
\[
	H(q_i,\,p_i)=T_2-T_0+\Pi=h
,\] 
где
\[
 T_2=\frac{1}{2}\sum_{j,\,k=1}^{m} a_{jk}\dot{q}_j\dot{q}_k,\qquad T_0=a_0
.\] 
\subsection{Канонические уравнения Гамильтона}
\begin{dfn}
	\emph{Каноническими уравнениями Гамильтона} называются уравнения
	\[
	\frac{dq_i}{dt}=\frac{\partial H}{\partial p_i},\qquad
	\frac{dp_i}{dt}=- \frac{\partial H}{\partial q_i}, \quad
	(i=1,\,2,\ldots,\n)
	.\] 
\end{dfn}
\subsection{Понятие первого интеграла динамической системы (не только для
гамильтоновых систем!)}\improvement{Изм.}
%\begin{dfn}
%	\emph{Первыми интегралами дифференциальных уравнений движения}
%	называются функции от координат точек и их скоростей, которые не
%	изменяются во время движения системы.
%\end{dfn}
\begin{dfn}
	 \emph{Первым интегралом дифференциальных уравнений движения}
	называется функция $I(\mathbf{q},\,\mathbf{p},\,t)$ отличная от постоянной,
	производная которой в силу уравнений движения
	\begin{gather*}
		\dot{\mathbf{q}}=\mathbf{f}_q (\mathbf{q},\,\mathbf{p},\,t),\\
		\dot{\mathbf{p}}=\mathbf{f}_p (\mathbf{q},\,\mathbf{p},\,t),
	\end{gather*}
	равна нулю:
	\[
	\frac{\partial I}{\partial \mathbf{q}}\mathbf{f}_q+
	\frac{\partial I}{\partial \mathbf{p}} \mathbf{f}_p+
	\frac{\partial I}{\partial t}=0 
	.\] 
\end{dfn}
\subsection{Понятие скобок Пуассона, их свойства (дистрибутивность,
антикоммутативность etc). Критерий первого интеграла гамильтоновой
системы}
\begin{dfn}
	Пусть $u$ и $v$ --- дважды непрерывно дифференцируемые функции от
	$q_1,\ldots,\,q_n,\,p_1,\ldots,\,p_n,t$. Выражение
	\[
		(u,\,v)= \sum_{i=1}^{n}\left( \frac{\partial u}{\partial q_i}
		\frac{\partial v}{\partial p_i} -\frac{\partial u}{\partial p_i} 
	\frac{\partial v}{\partial q_i} \right)
	\] 
	называют \emph{скобкой Пуассона} функций $u$ и $v$.
\end{dfn}
Свойства скобок Пуассона:
\begin{enumerate}
	\item $(u,\,v)=-(v,\,u)$,
	\item $(cu,\,v)=c(u,\,v) \quad (c=\mathrm{const})$,
	\item $(u+v,\,w)=(u,\,w)+(v,\,w)$,
	\item $\displaystyle \frac{\partial }{\partial t} (u,\,v)=
		\left( \frac{\partial u}{\partial t} ,\,v \right) +
		\left(u,\,\frac{\partial v}{\partial t}\right) $,\\
	$\displaystyle \frac{\partial }{\partial q_i} (u,\,v)=
		\left( \frac{\partial u}{\partial q_i} ,\,v \right) +
		\left(u,\,\frac{\partial v}{\partial q_i}\right) $,\\
	$\displaystyle \frac{\partial }{\partial p_i} (u,\,v)=
		\left( \frac{\partial u}{\partial p_i} ,\,v \right) +
		\left(u,\,\frac{\partial v}{\partial p_i}\right) $,\improvement{
		Добавлены производные по $q_i,\,p_i$}
	\item $((u,\,v),\,w)+((v,\,w),\,u)+((w,\,u),\,v)=0$.
\end{enumerate}
Необходимое и достаточное  условие того, что $f$ --- первый интеграл, можно
записать в виде равенства
\[
	\frac{\partial f}{\partial t} +(f,\,H)=0
.\] 
\subsection{Трубка прямых путей. Интегральные инварианты Пуанкаре и
Пуанкаре-Картана}
\begin{dfn}
	Множество прямых путей <<выпускаемых>> из замкнутого
	несамопересекающегося контура $C_0$ в $(2n+1)$-мерном расширенном
	фазовом пространстве $q,\,p,\,t$ динамической системой, движущейся
	в потенциальном поле и имеющей гамильтониан $H$ образует \emph{трубку
	прямых путей}.
\end{dfn}
\begin{dfn}
	Контурный интеграл
	\[
		J = \oint\limits_C \left(\sum p_j dq_j - Hdt\right),
	\] 
	 взятый по контуру $C$, охватывающему трубку прямых путей называют
	 \emph{интегральным инвариантом Пуанкаре-Картана}.
\end{dfn}
Данный интеграл, взятый по любому контуру $C^*$, охватывающему определённую
трубку прямых путей, не зависит от выбора этого контура на трубке.
\begin{dfn}
	Контурный интеграл, рассматриваемый только на <<одновременных>>
	контурах $\tilde{C}$, имеет  вид
	\[
		J_1 = \oint\limits_{\tilde{C}} \sum p_j dq_j=\mathrm{const}
	\]
	и называется \emph{универсальным интегральным инвариантом Пуанкаре}.
\end{dfn}
Особенность интегрального инварианта, взятого в такой форме, состоит в том, что
в подынтегральное выражение уже не входит гамильтониан, и следовательно, этот
интегральный инвариант оказывается \emph{одинаковым для всех динамических
систем}, движущихся в произвольных потенциальных полях.
\subsection{Теорема Лиувилля о сохранении фазового объема. Общая формула для
изменения фазового объема произвольной динамической системы (не только
гамильтоновой!)}
\begin{thm}[Лиувилля]
	Фазовый объём гамильтоновой системы \impr{Исп.}$V$ не зависит от $t$,
	т.\:е. является инвариантом движения.
\end{thm}
Пусть задана система обыкновенных дифференциальных уравнений 
$\dot{\mathbf{x}}=\mathbf{f}(\mathbf{x})$, $\mathbf{x}=(x_1,\ldots,\,x_n)$,
решения которой продолжаются на всю ось времени. Пусть также $D(t)$ ---
область в пространстве $\{\mathbf{x}\}$ и $v(t)$ --- её  объём, тогда
справедлива  следующая формула
\[
	\left.\frac{dv(t)}{dt}\right|_{t=t_0}= \int\limits_{D(t_0)}
		\operatorname{div}
		\mathbf{f}\,dx  \quad(dx=dx_1 \ldots dx_n)
.\] 

\subsection{Классификация интегральных инвариантов, теорема Ли Хуачжуна}
Универсальный относительный интегральный инвариант первого порядка в общем
виде можно записать так
\[
	\tilde{J}_1=\oint\limits_{\tilde{C}}\sum\left[ A_j(q,\,p,\,t)\delta q_j
	+B_j(q,\,p,\,t)\delta p_j\right] 
.\] 
\begin{thm}
	Любой универсальный относительный инвариант первого порядка $\tilde{J}_1$
	может отличаться от инварианта Пуанкаре лишь постоянным множителем,
	 т.\:е. для любого $J_1$ существует константа $c$
	 (одна и та же на всех трубках прямых путей рассматриваемой
	 гамильтоновой системы)\improvement{Добавлен комментарий} такая, что
	 \[
		 \tilde{J}_1=cJ_1
	 .\] 
\end{thm}
\subsection{Канонические преобразования. Производящие функции}
\label{subsec:tr}
%В некоторой области фазового пространства $q_1,\,q_2,\ldots,\,q_n,\,p_1,\,p_2,
%\ldots,\,p_n$ рассмотрим обратимое, дважды непрерывно дифференцируемое
%преобразование $\mathbf{q},\,\mathbf{p} \to \mathbf{Q},\, \mathbf{P}$,
%содержащую время $t$ в качестве параметра:
%\[
%	Q_i=Q_i(\mathbf{q},\,\mathbf{p},\,t),\quad P_i=P_i(\mathbf{q},\,\mathbf{p},\,
%	t),\quad (i=1,\,2,\ldots,\,n),
%\]
%\begin{dfn}
%	Преобразование выше называется \emph{каноническим}, если существует
%	такое постоянное число $c \neq 0$, что матрица Якоби
%	\[
%		\mathbf{M}= \begin{pmatrix} \frac{\partial \mathbf{Q}}{\partial \mathbf{q}} &
%		\frac{\partial \mathbf{Q}}{\partial \mathbf{p}} \\
%	\frac{\partial \mathbf{P}}{\partial \mathbf{q}} &
%\frac{\partial \mathbf{P}}{\partial \mathbf{p}} \end{pmatrix} 
%	\] 
%	удовлетворяет тождеству
%	\[
%	\mathbf{M}'\mathbf{JM}=c \mathbf{J},
%	\]
%	где матрица $\mathbf{J}$ определена равенством
%	\[
%		\mathbf{J}=\begin{pmatrix} 0 &   \mathbf{E}_n \\
%		-\mathbf{E}_n & 0\end{pmatrix} 
%	.\] 
%	Число $c$ называется \emph{валентностью канонического преобразования}.
%\end{dfn}
%\begin{thm}
%	Для каноничности рассматримаего выше преобразования необходимо и
%	достаточно, чтобы существовал отличная от науля постоянная $c$ такая,
%	что выражение
%	\[
%	c \sum_{k=1}^{n} p_k \delta q_k- \sum_{k=1}^{n} P_k \delta Q_k
%	\] 
%	является полным дифференциалом некоторой функции $F(\mathbf{q},\,
%	\mathbf{p},\,t)$.
%\end{thm}
%\begin{dfn}
%	Функцию $F(\mathbf{q},\,\mathbf{p},\,t)$ из предыдущей теоремы
%	называют \emph{производящей функцией}.
%\end{dfn}
Рассмотрим преобразование
\[
	q^*_j=\phi(q,\,p,\,t),\quad p^*_j=\psi(q,\,p,\,t)\quad (j=1,\ldots,\,n),
,\] 
которое переводит <<старые>> гамильтоновы переменные $p$ и $q$ в
<<новые>> гамильтоновы переменные $q^*$ и $p^*$ 
\begin{dfn}
	Преобразование выше  называется \emph{каноническим}, если оно переводит
	любую гамильтонову систему
	\[
	\frac{dq_j}{dt}=\frac{\partial H}{\partial p_j} , \quad
	\frac{dp_j}{dt}=-\frac{\partial H}{\partial q_j}  \quad
	(j=1,\ldots,\,n)
\]
в новую гамильтонову систему
\[
\frac{dq^*_j}{dt}=\frac{\partial H^*}{\partial p^*_j} ,\quad
\frac{dp^*_j}{dt}=-\frac{\partial H^*}{\partial q^*_j} \quad
(j=1,\ldots,\,n)
.\] 
\end{dfn}
\begin{thm}
	Для того чтобы рассматриваемое преобразование было каноническим,
	необходимо и достаточно, чтобы существовали такая функция 
	$F(q,\,p,\,t)$ и такое число $c$, чтобы тождественно выполнялось
	равенство
	\[
		\sum \psi_j \delta \phi_j -c \sum p_j \delta q_j=-\delta F(q,\,
		p,\,t)
	.\] 
\end{thm}
В последней формуле $\delta$ --- оператор дифференцирования функции от
$q,\,p,\,t$ при <<замороженном>> времени, т.\:е.
\[
\delta \phi_j =d \phi_j - \frac{\partial \phi_j}{\partial t} dt,\quad
\delta F=dF-\frac{\partial F}{\partial t} dt.
.\]
\begin{dfn}
	Функцию $F(q,\,p,\,t)$ из последней теоремы называют \emph{производящей}.
\end{dfn}
\subsection{Замена гамильтониана при каноническом преобразовании, $(q,\,p)$ 
описание}
\begin{thm}
Пусть преобразование из предыдущего пункта является каноническим, причём
$c$ и  $F(q,\,p,\,t)$, при которых удовлетворяется тождество последней
теоремы известны. Тогда <<новый>>  гамильтониан $H^*$ определяется по
<<старому>> гамильтониану $H$, если в функции
\[
H^*=cH+\frac{\partial F}{\partial t} +\sum \psi_j \frac{\partial \phi_j}{\partial t} 
\]
выразить переменные $q$ и $p$ через  $q^*$ и $p^*$ при помощи обратных
преобразований (если таковые существуют).
\end{thm}
\subsection{Свободные  преобразования, $(q,\,q^*)$ описание. Формулы
	преобразования
импульсов гамильтониана}
Преобразование из пункта \ref{subsec:tr} называется \emph{свободным}, если
для первых $n$ его уравнений
\[
	q^*_j=\phi_j(q,\,p,\,t) \quad (j=1,\ldots,\,n)
\]
якобиан отличен от нуля:
\[
\det \left\|\frac{\partial \phi}{\partial p} \right\|=
\begin{vmatrix}
	\frac{\partial \phi_1}{\partial p_1}  & \hdots & \frac{\partial \phi_1}{\partial p_n} \\
	\vdots & & \vdots\\
	\frac{\partial \phi_n}{\partial 1} & \hdots & \frac{\partial \phi_n}{\partial p_n} 
\end{vmatrix} \neq 0
.\] 
Тогда для функции $S(q,\,q^*,\,t)=F(q,\,p,\,t)$ можем записать следующие
соотношения
 \[
\frac{\partial S}{\partial q_j} =c p_j, \qquad \frac{\partial S}{\partial q_j^*} =
-p_j^* \quad (j=1,\ldots,\,n)
,\] 
и, кроме того, равенство связывающее <<старый>> и <<новый>> гамильтонианы:
\[
H^*=cH+\frac{\partial s}{\partial t} 
.\] 
\subsection{\emph{<<Наивная>> теория возмущений, использование $(q,\,p^*)$ описания
для задания преобразований, близких к тождественным. Метод Биркгофа, понятие
резонанса}}
🤔🤔🤔
\subsection{Уравнение Гамильтона-Якоби. Полный интеграл уравнения
Гамильтона-Якоби}
\begin{dfn}
	Уравнением Гамильтона-Якоби называют следующее выражение
\[
	\frac{\partial S}{\partial t} +H\left(q,\,\frac{\partial S}{\partial q} ,\,t\right)=0
.\]
\end{dfn}
\begin{dfn}
Любая функция $S(q,\,\alpha,\,t)$, обращающая уравнение Гамильтона-Якоби в
тождество, зависящая от $n$ констант $\alpha$ и удовлетворяющая условию
\[
\det \left\| \frac{\partial^2 S}{\partial q_k \partial \alpha_j} \right\|^n_{k,\,j=1}\neq 0
\]
называется \emph{полным интегралом} уравнения Гамильтона-Якоби.
\end{dfn}
\subsection{Понятие адиабатических инвариантов динамических систем}
Пусть $H(p,\,q;\,\lambda)$ --- фиксированная дважды непрерывно дифференцируемая
функция $\lambda$. Положим $\lambda=\epsilon t$ и будем рассматривать
полученную систему с медленно меняющимся параметром $\lambda=\epsilon t$:
\[
	\dot{p}=-\frac{\partial H}{\partial q} , \quad
	\dot{q}=\frac{\partial H}{\partial p} ,\quad H=H(p,\,q;\,\epsilon t)
	\tag{*}
	\label{eq:1}
.\] 
\begin{dfn}
	Величина $I(p,\,q;\,\lambda)$ называется \emph{адиабатическим 
	инвариантом} системы (\ref{eq:1}), если для всякого $\kappa>0$
	существует   $\epsilon_0>0$ такое, что если  $0<\epsilon<\epsilon_0$,
	$0<t<1 /\epsilon$, то
	 \[
		 |I(p(t),\,q(t);\,\epsilon t)-I(p(0),\,q(0);\,0)|<\kappa
	.\] 
\end{dfn}
\subsection{Переменные действие угол. Условие возможности перехода к ним,
формулы
перехода (случай одной степени свободы)}
hi
\subsection{Понятие интегрируемых гамильтоновых систем. Теорема
Лиувилля-Арнольда}
hi
\subsection{Резонансные и нерезонансные торы}
hi
\subsection{\emph{Невырожденность и изоэнергетическая невырожденность
гамильтониана}}
hi
\subsection{\emph{КАМ-теорема}}
hi
\subsection{\emph{Важные следствия КАМ-теоремы для систем с двумя степенями свободы,
обладающих свойством изоэнергетической и обычной невырожденности}}
hi
\subsection{\emph{Понятие детерминированного хаоса в динамических системах}}
hi
\subsection{\emph{Сечения Пуанкаре}}
hi
\subsection{\emph{Фрактальная размерность}}
hi
\end{document} % конец документа
