
\documentclass[a4paper,12pt]{article}

% Этот шаблон документа разработан в 2014 году
% Данилом Фёдоровых (danil@fedorovykh.ru) 
% для использования в курсе 
% <<Документы и презентации в \LaTeX>>, записанном НИУ ВШЭ
% для Coursera.org: http://coursera.org/course/latex .
% Исходная версия шаблона --- 
% https://www.writelatex.com/coursera/latex/5.3

% В этом документе преамбула

%%% Работа с русским языком
\usepackage{cmap}					% поиск в PDF
\usepackage{mathtext} 				% русские буквы в формулах
\usepackage[T2A]{fontenc}			% кодировка
\usepackage[utf8x]{inputenc}			% кодировка исходного текста
\usepackage[english,russian]{babel}	% локализация и переносы
\usepackage{indentfirst}
\frenchspacing

\renewcommand{\epsilon}{\ensuremath{\varepsilon}}
\renewcommand{\phi}{\ensuremath{\varphi}}
\renewcommand{\kappa}{\ensuremath{\varkappa}}
\renewcommand{\le}{\ensuremath{\leqslant}}
\renewcommand{\leq}{\ensuremath{\leqslant}}
\renewcommand{\ge}{\ensuremath{\geqslant}}
\renewcommand{\geq}{\ensuremath{\geqslant}}
\renewcommand{\emptyset}{\varnothing}

%%% Дополнительная работа с математикой
\usepackage{amsmath,amsfonts,amssymb,amsthm,mathtools} % AMS
\usepackage{icomma} % "Умная" запятая: $0,2$ --- число, $0, 2$ --- перечисление

%% Номера формул
%\mathtoolsset{showonlyrefs=true} % Показывать номера только у тех формул, на которые есть \eqref{} в тексте.
%\usepackage{leqno} % Нумереация формул слева

%% Свои команды
\DeclareMathOperator{\sgn}{\mathop{sgn}}

%% Перенос знаков в формулах (по Львовскому)
\newcommand*{\hm}[1]{#1\nobreak\discretionary{}
{\hbox{$\mathsurround=0pt #1$}}{}}

%%% Работа с картинками
\usepackage{graphicx}  % Для вставки рисунков
\graphicspath{{figures/}}  % папки с картинками
\setlength\fboxsep{3pt} % Отступ рамки \fbox{} от рисунка
\setlength\fboxrule{1pt} % Толщина линий рамки \fbox{}
\usepackage{wrapfig} % Обтекание рисунков текстом

%%% Работа с таблицами
\usepackage{array,tabularx,tabulary,booktabs} % Дополнительная работа с таблицами
\usepackage{longtable}  % Длинные таблицы
\usepackage{multirow} % Слияние строк в таблице

%%% Теоремы
\theoremstyle{plain} % Это стиль по умолчанию, его можно не переопределять.
\newtheorem{theorem}{Теорема}
\newtheorem*{thm}{Теорема}
\newtheorem{proposition}[theorem]{Утверждение}
 
\theoremstyle{definition} % "Определение"
\newtheorem{corollary}{Следствие}[theorem]
\newtheorem*{dfn}{Определение}
\newtheorem{problem}{Задача}
\newtheorem*{problem*}{Задача}

 
\theoremstyle{remark} % "Примечание"
\newtheorem*{nonum}{Решение}

%%% Программирование
\usepackage{etoolbox} % логические операторы

%%% Страница
%\usepackage{extsizes} % Возможность сделать 14-й шрифт
%\usepackage{geometry} % Простой способ задавать поля
%	\geometry{top=25mm}
%	\geometry{bottom=35mm}
%	\geometry{left=35mm}
%	\geometry{right=20mm}
 
\usepackage{fancyhdr} % Колонтитулы
%	\pagestyle{fancy}
 %	\renewcommand{\headrulewidth}{0pt}  % Толщина линейки, отчеркивающей верхний колонтитул
	%\lfoot{Нижний левый}
	%\rfoot{Нижний правый}
	%\rhead{Верхний правый}
	%\chead{Верхний в центре}
	%\lhead{Верхний левый}
	%\cfoot{Нижний в центре} % По умолчанию здесь номер страницы

\usepackage{setspace} % Интерлиньяж
%\onehalfspacing % Интерлиньяж 1.5
%\doublespacing % Интерлиньяж 2
%\singlespacing % Интерлиньяж 1

\usepackage{lastpage} % Узнать, сколько всего страниц в документе.

\usepackage{soul} % Модификаторы начертания

\usepackage{hyperref}
\usepackage[usenames,dvipsnames,svgnames,table,rgb,pdftex,dvipsnames]{xcolor}
\hypersetup{				% Гиперссылки
    unicode=true,           % русские буквы в раздела PDF
    pdftitle={Заголовок},   % Заголовок
    pdfauthor={Автор},      % Автор
    pdfsubject={Тема},      % Тема
    pdfcreator={Создатель}, % Создатель
    pdfproducer={Производитель}, % Производитель
    pdfkeywords={keyword1} {key2} {key3}, % Ключевые слова
    colorlinks=true,       	% false: ссылки в рамках; true: цветные ссылки
    linkcolor=red,          % внутренние ссылки
    citecolor=black,        % на библиографию
    filecolor=magenta,      % на файлы
    urlcolor=cyan           % на URL
}

\usepackage{csquotes} % Еще инструменты для ссылок

%\usepackage[style=apa,maxcitenames=2,backend=biber,sorting=nty]{biblatex}

\usepackage{multicol} % Несколько колонок

\usepackage{tikz} % Работа с графикой
\usepackage{pgfplots}
\usepackage{pgfplotstable}
\usepackage{coloremoji}
\usepackage{floatrow}
\usepackage{subcaption}
\newcommand*{\N}{\mathbb{N}}
\newcommand*{\R}{\mathbb{R}}
\newcommand*{\K}{\mathbb{K}}
\newcommand*{\V}{\mathcal{V}}
\newcommand*{\A}{\mathcal{A}}
\newcommand*{\ii}{\mathbf{1}}
\newcommand*{\oo}{\mathbf{0}}
\newcommand*{\ba}{\mathbf{a}}
\newcommand*{\bb}{\mathbf{b}}
\newcommand*{\Q}{\mathbb{Q}}
\graphicspath{{images/}}
%\usepackage{breqn}

\renewcommand\thesubfigure{\asbuk{subfigure}}
%\addbibresource{master.bib}

\usepackage{import}
\usepackage{pdfpages}
\usepackage{transparent}
\usepackage{xcolor}
\usepackage{xifthen}

%\newcommand{\incfig}[1]{%
%    \def\svgwidth{\columnwidth}
%    \import{./figures/}{#1.pdf_tex}
%}


\newcommand{\incfig}[2][1]{%
    \def\svgwidth{#1\columnwidth}
    \import{./figures/}{#2.pdf_tex}
}
\usepackage{titlesec}
\titleformat{\section}{\normalfont\Large\bfseries}{}{0pt}{}


\pdfsuppresswarningpagegroup=1
\pgfplotsset{compat=1.16}
\renewcommand{\thesection}{}
\renewcommand{\thesubsection}{\arabic{subsection}}

\usepackage{marginnote}
\usepackage{xargs}
\usepackage[colorinlistoftodos,prependcaption,shadow, textwidth=3cm]{todonotes}
\newcommandx{\improvement}[2][1=]{\todo[disable,linecolor=OliveGreen,backgroundcolor=OliveGreen!25,bordercolor=OliveGreen,#1]{#2}}
\newcommandx{\unsure}[2][1=]{\todo[disable,linecolor=YellowOrange,backgroundcolor=YellowOrange!25,bordercolor=YellowOrange,#1]{#2}}
\newcommandx{\didntdone}[2][1=]{\todo[disable,linecolor=red,backgroundcolor=red!25,bordercolor=red,#1]{#2}}
\newcommandx{\impr}[2][1=]{\todo[linecolor=OliveGreen,backgroundcolor=OliveGreen!25,bordercolor=OliveGreen,#1]{#2}}
%%\setcounter{secnumdepth}{0}
%%% Local Variables:
%%% mode: latex
%%% TeX-master: "master"
%%% End:


\title{Теорминимум \\
по теормеху}
%\author{Драчов Ярослав\\ Факультет общей и прикладной физики, МФТИ}


\begin{document} % конец преамбулы, начало документа
%\input{titlepage}
%\pdfsuppresswarningpagegroup=1a
\maketitle
\section*{Первое задание}
\subsection{Определение положения равновесия}
hi
\subsection{Устойчивое положение равновесия}
hi
\subsection{Теорема Лагранжа-Дирихле}
hi
\subsection{Первая теорема Ляпунова о неустойчивости}
hi
\subsection{Вторая теорема Ляпунова о неустойчивости}
hi
\subsection{Нормальные координаты}
hi
\subsection{Асимптотически устойчивое положение равновесия}
hi
\subsection{Теорема Ляпунова о лианеризованных системах}
hi
\subsection{Критерии Рауса-Гурвица и Льенара-Шипара устойчивости многочлена}
hi
\subsection{Теорема Ляпунова об устойчивости/асимптотической устойчивости
(функция Ляпунова)}
hi
\subsection{Теорема Барбашина-Красовского}
hi
\subsection{Теорема Четаева о неустойчивости}
hi
\subsection{Понятие о бифуркации положений равновесия}
hi
\subsection{Бифуркация Андронова-Хопфа}
hi
\subsection{Метод Биркгофа(?) приведения к нормальной форме. Понятие резонанса}
hi
\subsection{Вынужденные колебания под действием периодической силы. Частотная
характеристика, амплитудно-фазовая характеристика. Необходимые и достаточные
условия возникновения резонанса в таких системах.}
hi
\section*{Второе задание}
\setcounter{subsection}{0}
\subsection{Понятие краевой задачи для лагранжевых систем, теорема Гамильтона-
hi
Остроградского, формула изменения лагранжиана при замене координат и
времени, теорема Нётер}
\subsection{Переменные Гамильтона. Обобщенные импульсы}
hi
\subsection{Функция Гамильтона через функцию Лагранжа}
hi
\subsection{Функция Гамильтона через обобщенный потенциал и кин.энергию (случай
обобщенно консервативной системы) (см. 286-287 стр. Маркеева)}
hi
\subsection{Канонические уравнения Гамильтона}
hi
\subsection{Понятие первого интеграла динамической системы (не только для
гамильтоновых систем!)}
hi
\subsection{Понятие скобок Пуассона, их свойства (дистрибутивность,
антикоммутативность etc). Критерий первого интеграла гамильтоновой
системы}
hi
\subsection{Трубка прямых путей. Интегральные инварианты Пуанкаре и Пуанкаре-
Картана}
hi
\subsection{Теорема Лиувилля о сохранении фазового объема. Общая формула для
изменения фазового объема произвольной динамической системы (не только
гамильтоновой!)}
hi
\subsection{Классификация интегральных инвариантов, теорема Ли Хуачжуна}
hi
\subsection{Канонические преобразования. Производящие функции}
hi
\subsection{Замена гамильтониана при каноническом преобразовании, $(q,\,p)$ 
описание}
hi
\subsection{Свободные  преобразования, $(q,\,q^*)$ описание. Формулы
	преобразования
импульсов гамильтониана}
hi
\subsection{\emph{<<Наивная>> теория возмущений, использование $(q,\,p^*)$ описания
для задания преобразований, близких к тождественным. Метод Биркгофа, понятие
резонанса}}
hi
\subsection{Уравнение Гамильтона-Якоби. Полный интеграл уравнения
Гамильтона-Якоби}
hi
\subsection{Понятие адиабатических инвариантов динамических систем}
hi
\subsection{Переменные действие угол. Условие возможности перехода к ним,
формулы
перехода (случай одной степени свободы)}
hi
\subsection{Понятие интегрируемых гамильтоновых систем. Теорема
Лиувилля-Арнольда}
hi
\subsection{Резонансные и нерезонансные торы}
hi
\subsection{\emph{Невырожденность и изоэнергетическая невырожденность
гамильтониана}}
hi
\subsection{\emph{КАМ-теорема}}
hi
\subsection{\emph{Важные следствия КАМ-теоремы для систем с двумя степенями свободы,
обладающих свойством изоэнергетической и обычной невырожденности}}
hi
\subsection{\emph{Понятие детерминированного хаоса в динамических системах}}
hi
\subsection{\emph{Сечения Пуанкаре}}
hi
\subsection{\emph{Фрактальная размерность}}
hi
\end{document} % конец документа
